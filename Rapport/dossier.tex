\documentclass[11pt]{report}

\usepackage[frenchb]{babel}
\usepackage[babel=true]{csquotes} % pour les guillemets (csquotes utilise la langue definie dans babel)
\usepackage[T1]{fontenc}
\usepackage[utf8]{inputenc}
\usepackage{a4wide}
\usepackage{graphicx}
\usepackage{float} % pour forcer la position des images (H)
\usepackage{textcomp}
\usepackage{amsmath} %pour les maths
\usepackage{fixltx2e} %pour les indices des lettres 
\usepackage{gensymb}
\usepackage{caption}
\usepackage{subcaption}
%\usepackage{xcolor}
\usepackage[table]{xcolor}
\usepackage{array,multirow,makecell}
\setcellgapes{1pt}
\usepackage[a4paper]{geometry}%pour les marges
\geometry{hscale=0.75,vscale=0.85,centering}%pour les marges
\usepackage[final]{pdfpages} %insertion des pdf
\usepackage{listings}
\usepackage{slashbox}% pour les "diagonales" dans cell tableau
\usepackage{lscape}%pour le mode paysage



\setlength{\parskip}{1ex plus .4ex minus .4ex} % espacement paragraphes

\makeatletter
	\renewcommand{\fnum@figure}{\textsc{Fig.}~\thefigure} %"Fig." à la place de "Figure"
\makeatother

% configuration text dans tableaux
\newcolumntype{R}[1]{>{\raggedleft\arraybackslash }b{#1}}
\newcolumntype{L}[1]{>{\raggedright\arraybackslash }b{#1}}
\newcolumntype{C}[1]{>{\centering\arraybackslash }b{#1}}


% configuration des listings
%\input{../../../Divers/listing.tex}
%to insert file code
\lstset
{
	language=C,               		% choose the language of the code
	numbers=left,                   % where to put the line-numbers
	stepnumber=1,                   % the step between two line-numbers.        
	numbersep=5pt,                  % how far the line-numbers are from the code
	backgroundcolor=\color{white},  % choose the background color. You must add \usepackage{color}
	showspaces=false,               % show spaces adding particular underscores
	showstringspaces=false,         % underline spaces within strings
	showtabs=false,                 % show tabs within strings adding particular underscores
	tabsize=2,                      % sets default tabsize to 2 spaces
	captionpos=b,                   % sets the caption-position to bottom
	breaklines=true,                % sets automatic line breaking
	breakatwhitespace=true,         % sets if automatic breaks should only happen at whitespace
	title=heapsort.c,                 % show the filename of files included with \lstinputlisting;
	%title=\lstname,                 % show the filename of files included with \lstinputlisting;
	frame=lrb,
	xleftmargin=\fboxsep,			% pour le cadre
	xrightmargin=-\fboxsep,			% pour le cadre
	basicstyle=\ttfamily,
	keywordstyle=\color{blue}\ttfamily,
	stringstyle=\color{red}\ttfamily,
	commentstyle=\color{green}\ttfamily,
	morecomment=[l][\color{magenta}]{\#}
}

\begin{document}

\begin{titlepage}
\parindent=0pt
\begin{minipage}{0.4\textwidth}
	\begin{flushleft} \large
		\textsc{Ahallouch} Kaoutare M18\\[.2cm]
		\textsc{MASTROLLILI} Bob M18\\[.2cm]
		\textsc{SOULEIMAN} Ilsan M18\\[.2cm]
	\end{flushleft}
\end{minipage}
\hspace*{\stretch{1}}
\vspace*{\stretch{1}}
\begin{center}
\includegraphics[scale= 1]{images/logo.jpg}%
\end{center}
\vspace*{\stretch{1}}
\hrulefill
\begin{center}\bfseries\Huge
    Création d’un outil d’analyse de code (God class) 
\end{center}
\hrulefill
\vspace*{1cm}
\begin{center}\bfseries\Large
	UE: Génie logiciel et conduite de projets informatiques  2\\
	(\textit{AA: Techniques et conduite des tests logiciels})
\end{center}
  
\vspace*{\stretch{2}}
\begin{center}
       Date de rentrée du rapport : \today
\end{center}   

\vspace*{\stretch{6}}
\begin{center}\bfseries\Large
Année académique : 2014-2015
\end{center}
\end{titlepage}

\tableofcontents


\chapter{Définitions}

	\section{God Class}
	Avant de débuter ce travail, il parait nécessaire de définir ce qu'on entend par \textit{God Class}. Les \textit{God Class} sont des classes qui ont trop de responsabilités ce qui va à l'encontre de la programmation orientée objet. Les \textit{God Class} sont caractérisées par un nombre important de méthodes, d'appels à des méthodes extérieures ou d'accès à des variables membres. Afin de mettre en évidence ce type de classes, différents indicateurs peuvent être utilisés. 
	
	\section{Indicateurs de God Class}
	Nous avons décidés d'utiliser principalement trois indicateurs: le WMC, le TCC et l'AFTD qui seront explicités d'avantage ci-dessous.   
		
		\subsection{WMC}
		Le \textit{Weighted Method Count} (WMC) est un métrique qui permet de mesurer la complexité d'une classe. Pour ce faire, ce critère peut se baser notamment sur le nombre de méthodes d'une classe ou  la complexité cyclomatique de MCCabe peut être utilisée. Pour cet indicateur, une valeur faible est préconiséee. Mais qu'entend-on par \textit{faible} ? Dans la littérature, certains auteurs considèrent comme faible une valeur entre 20 et 50 d'autres s'accordent sur une valeur de 10. Mais quoiqu'il en soit, une valeur supérieure à 50 est à éviter. En effet, la maintenance d'une classe ayant plus de 50 méthodes est très difficile ainsi que sa lisibilité et son refactoring.   
		
		\subsection{TCC}
		Le \textit{Tight Class Cohesion} (TCC) est un métrique qui permet de mesurer la cohésion d'une classe. Cet indicateur s'appuie sur l'analyse des relation entre méthodes. Il se situe généralement entre 0 et 1. La détection d'une classe avec peu de cohésion permettra, par exemple, de déterminer si une classe à besoin ou non d'être restructurée en deux autres classes.  Pour cet indicateur, une haute valeur est préférable.  
		
		\subsection{AFTD}
		L'\textit{Acces to Foreign Data} (AFTD) est un indiacteur qui permet d'analyser les accès à des méthodes extérieures par voies indirectes ou directes. Il s'agit d'un indicateur qui renseigne sur le degré d'encapsulation d'une classe. 
		
	
\chapter{Consignes}
	
	\section{Enoncé}
	\fcolorbox{lightgray}{teal}
	{
		\begin{minipage}{1\textwidth}
			\textit
			{
				\subsubsection{Consignes générales}
				\begin{itemize}
					\item Travail a effectué par groupe de 3. Chaque groupe choisit un thème. Pour chaque thème, les
					explications complémentaires nécessaires pour la réalisation du travail vous seront fournies.
					\item Utilisation d’un serveur d’intégration
					\item Utilisation d’au moins une technique de tests en boîte noire et une technique de tests en
					boîte blanche (vue au cours ou autre)
					\item Remise d’un rapport explicitant les objectifs fixés pour les tests, la ou les méthodologie(s)
					utilisée(s) ainsi que les résultats obtenus.
					\item Prises d’initiatives requises pour la réalisation du projet en respectant toutefois les quelques
					contraintes imposées !
					\\
				\end{itemize}
			}
		\end{minipage}
	}
	
	\fcolorbox{lightgray}{teal}
	{
		\begin{minipage}{1\textwidth}
			\textit
			{
				\subsubsection{Thème 2 : Création d’un outil d’analyse de code}
				Ecrire une application permettant de détecter des classes qualifiées de "God class". Une telle classe a
				trop de responsabilités et accède généralement directement aux données d’autres classes. La
				réutilisabilité et la facilité de compréhension de cette classe n’est dès lors pas facile. Une bonne
				pratique de programmation orientée objet veut que l’on attribue au plus une seule responsabilité à
				une classe. Les métriques utilisés pour détecter une "God class" sont (cf. ouvrage Object-oriented
				Metrics in Practice de Stéphane Ducasse et Michèle Lanza)\\
				\begin{itemize}
					\item Mesure de la complexité de la classe : Weighted Method Count (WMC)
					http://www.aivosto.com/project/help/pm-oo-ck.html
					\begin{enumerate}
						\item Le nombre de méthodes dans une classe ou
						\item La somme des complexités statiques de toutes les méthodes d’une classe. Dans ce
						dernier cas, la complexité cyclomatique de McCabe peut être utilisée comme mesure
						de la complexité.
					\end{enumerate}
					\item Mesure de la cohésion d’une classe : Tight Class Cohesion (TCC) : TCC)
					http://www.aivosto.com/project/help/pm-oo-cohesion.html
					Le métrique sert à évaluer la cohésion d’une classe. Sa valeur est située en 0 et 1. Il
					correspond au nombre relatif de pairs de méthodes d’une classe qui accède au moins
					à un même attribut de cette classe. En dessous de 1/3, on considérera que la classe a
					peu de cohésion. Votre solution doit permettre l’utilisation d’un autre métrique de
					cohésion.
					\item Indicateur du respect de l’encapsulation des autres classes : Access to Foreign Data (ATFD) :
					représente le nombre de classes extérieures auxquelles la classe accède les attributs (de
					manière directe ou par les accesseurs). Pas plus que quelques-uns (à définir) comme limite.\\
				\end{itemize}
				Les classes que l’on cherche à détecter dans un projet pour lequel on se propose d’étudier le
				code sont donc celles qui sont trop complexes, avec peu de cohésion et qui accèdent
				directement aux variables membres des autres classes. Remarque : cela ne signifie pas que la
				classe détectée est mal conçue mais simplement qu’il y a là un indicateur qui nous invite à
				pousser la réflexion quant à la conception du code.
			}
		\end{minipage}
	}


	\section{Outils utilisés}
	Pour l'élaboration de ce travail, nous avons , d'une part, utilisé un serveur d'intégration continue Jenkins mais également les librairies JUnit et Mockito. Pour les exemples d'utilisation de ces outils Cfr. le chapitre \ref{impl} 

\chapter{Méthodologie de travail}
Avant de débuter le projet et la partie implémentation, une phase de recherche a débuté. Pour optimiser notre temps de travail, nous nous sommes répartis les recherches. Ces dernières concernaient essentiellement les outils (Jenkins, JUnit, Mockito)  et les concepts théoriques (indicateurs, doublures ...) que nous avons été amenés à utiliser. Chaque membre du groupe a ainsi acquis le rôle de référant dans son domaine de recherche.\\ 

Une fois la phase de recherche terminée, la phase de mise en commun a alors pu débuter. Il était impérativement nécessaire que chaque membre du groupe est les mêmes connaissances concernant tous les concepts théoriques. De plus, cette phase a permis à chacun de donner son point de vue sur la manière à suivre.\\ 

Enfin, une fois cette phase de mise en commun passée, la phase d'implémentation à proprement dit a débutée. Chaque membre du groupe a implémenté un indicateur et les tests correspondants. L'implémentation s'est faite en parrallè. L'intégration de toutes les unités de travail au sein d'un seul et même projet s'est faite de manière assez aisée étant donné que la marche à suivre a été décidée au départ. \\ 

\chapter{Implémentation}\label{impl}

	Le projet se pésente sous la forme de quatre packages; un package pour chaque indicateur et un package pour le gui. Les trois packages concernant les indicateurs ont pu être implémentés de manière indépendante étant donné que l'objet passé en paramètre est une \textit{japa.parser.ast.CompilationUnit}. cette manière de procéder permet de ne parser qu'une seule fois le code de la classe. En effet, le résultat du parsing est une \textit{japa.parser.ast.CompilationUnit}. Tous les caluls de métriques s'effectuent à partir de cet objet. Cette indépendance au niveau du code des package a permis à chacun d'effectuer ses tests unitaires de manière isolée.
	
	\section{Package wmc}
		Les méthodes qui sont testées sont les méthodes de la classe \textit{WMCAnalyseur} et de la classe \textit{WMCCalculateur}. Pour les premières méthodes testées, il s'agit d'une analyse par valeurs limites. Pour les intervalles dont il est question dans la figure \ref{int}, trois valeurs sont testées; les valeurs aux extrêmités et une valeur au centre de l'intervalle. Ce qui donne en tout 9 tests; trois par chaque intervalle. En outre, trois autres tests sont effectués à l'aide mocks. Le test de la figure \ref{mock} permet de tester que la méthode \textit{getResult} appelle bien la méthode \textit{getMetrique}. Ce qui l'utilité première d'un mock (vérification des appels entre méthodes). La figure \ref{init_mock} montre également l'initialisation avec un mock aus sein de la classe de test. 
		
		\subsection{Extraits de code}
		
		\begin{figure}[h]
			\centering
			\boxed{\includegraphics[scale=1]{images/int.PNG}}
			\caption{Intervalles de classification pour le WMC}
			\label{int}
		\end{figure}
		
		\begin{figure}[h]
			\centering
			\boxed{\includegraphics[scale=1]{images/init_mock.PNG}}
			\caption{Initialisation d'un mock}
			\label{mock}
		\end{figure}
		
		\begin{figure}[h]
			\centering
			\boxed{\includegraphics[scale=1]{images/mock.PNG}}
			\caption{Exemple d'utilisation d'un mock}
			\label{mock}
		\end{figure}
		
	\newpage
	\section{Package aftd}
		Concernant ce package les tests effectués vont par paires. En effet, pour chaque méthode de la classe \textit{MeusureEncapsulationAFTD} testée il y a à chaque fois un test qui donne une valeur correcte et un autre qui donne une erreur.
		
		\newpage 
		\subsection{Extrait de code}
		\begin{figure}[h]
			\centering
			\boxed{\includegraphics[scale=0.9]{images/aftd.PNG}}
			\caption{Paire de test de la méthode \textit{meusurer()}}
			\label{aftd}
		\end{figure}
			
	%\section{Package tcc}
		%\subsection{Extrait de code}
			
	\newpage	
	\section{Package gui}
		Une God Class est détectée si les trois indicateurs indiquent su-imultanément qu'il s'agit d'une God Class. Le résultat est affiché au niveau du gui. Un exemple d'application du gui est donné à la figure \ref{gui}.
		\begin{figure}[h]
			\centering
			\boxed{\includegraphics[scale=0.9]{images/gui.PNG}}
			\caption{Exemple d'application}
			\label{gui}
		\end{figure}
	
		
	\section{Diagrammes UML}\label{uml}
		
		\begin{figure}[h]
			\centering
			\boxed{\includegraphics[scale=1, angle=90]{images/uml1.PNG}}
			\caption{Diagramme UML du package wmc}
			\label{uml1}
		\end{figure}
		
		\begin{figure}[h]
			\centering
			\boxed{\includegraphics[scale=0.7, angle=90]{images/uml2.PNG}}
			\caption{Diagramme UML des packages aftd et tcc}
			\label{uml2}
		\end{figure}
		

\chapter{Conclusion}
	Avec l'expérience de programmation que nous avons acquis jusqu'à présent, nous avons toujours eu l'habitude d'écrire le code avant d'écrire des tests pour tester notre code. La plupart du temps, nous n'avions pas recours à des librairies comme JUnit qui génèrent la structure des tests à effectuer. Nous effections les tests au sein d'un main. Cependant, pour l'élaboration de ce travail, il nous a été demandé de fonctionner en TDD (\textit{Test Driven Development}). Si, en soit, la théorie à sujet paraissait claire l'adaptée en pratique a été assez difficile dans un premier temps. En effet, cela allait à l'encontre de notre façon de programmer habituelle. Néanmoins, après moult recherches sur le sujet, nous avons enfin résussi à appréhender cette nouvelle façon de programmer. En fait, le TDD consistait à partir d'une méthode vide et à la compléter en fonction des résultats attendus et implémentés au niveaux des tests. Plus le nombre de tests augmentait plus la méthode était étoffée. Cette manière de procéder évite grandement les erreurs de programmation, puisqu'on focntionne étape par étape et qu'en cas d'erreur nous savons localiser aisément l'erreur. C'est là un des avantages du TDD.     

\chapter{Annexes}
		\includepdf{images/cahier.pdf}
		\includepdf[pages=1-5]{images/doc.pdf}
		
\end{document} 